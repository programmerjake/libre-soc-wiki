\documentclass[slidestop]{beamer}
\usepackage{beamerthemesplit}
\usepackage{graphics}
\usepackage{pstricks}

\title{Simple-V RISC-V Extension for Vectorisation and SIMD}
\author{Luke Kenneth Casson Leighton}


\begin{document}

\frame{
   \begin{center}
    \huge{Simple-V RISC-V Extension for Vectors and SIMD}\\
    \vspace{32pt}
    \Large{Flexible Vectorisation}\\
    \Large{(aka not so Simple-V?)}\\
    \vspace{24pt}
    \Large{Chennai 9th RISC-V Workshop}\\
    \vspace{24pt}
    \large{\today}
  \end{center}
}

\frame{\frametitle{Why another Vector Extension?}

 \begin{itemize}
   \item RVV very heavy-duty (excellent for supercomputing)\vspace{10pt}
   \item Simple-V abstracts parallelism (based on best of RVV)\vspace{10pt}
   \item Graded levels: hardware or software-emulation\vspace{10pt}
   \item Even Compressed instructions become vectorised\vspace{10pt}
  \end{itemize}
  What Simple-V is not:\vspace{10pt}
   \begin{itemize}
   \item A full supercomputer-level Vector Proposal\vspace{10pt}
   \item A replacement for RVV (designed to be augmented)\vspace{10pt}
  \end{itemize}
}

\frame{\frametitle{Quick refresher on SIMD}

 \begin{itemize}
   \item SIMD very easy to implement (and very seductive)\vspace{10pt}
   \item Parallelism is in the ALU\vspace{10pt}
   \item Zero-to-Negligeable impact for rest of core\vspace{10pt}
  \end{itemize}
  Where SIMD Goes Wrong:\vspace{10pt}
   \begin{itemize}
   \item See "Why SIMD considered harmful"\vspace{10pt}
   \item (Corner-cases alone are extremely complex)\vspace{10pt}
   \item O($N^{6}$) ISA opcode proliferation!\vspace{10pt}
  \end{itemize}
}

\frame{\frametitle{Quick refresher on RVV}

 \begin{itemize}
   \item Extremely powerful (extensible to 256 registers)\vspace{10pt}
   \item Supports polymorphism, several datatypes (inc. FP16)\vspace{10pt}
   \item Requires a separate Register File\vspace{10pt}
   \item Can be implemented as a separate pipeline\vspace{10pt}
  \end{itemize}
  However...\vspace{10pt}
   \begin{itemize}
   \item 98 percent opcode duplication with rest of RV (CLIP)\vspace{10pt}
   \item Extending RVV requires customisation\vspace{10pt}
  \end{itemize}
}


\frame{\frametitle{How is Parallelism abstracted?}

 \begin{itemize}
   \item Almost all opcodes removed in favour of implicit "typing"\vspace{10pt}
   \item Primarily at the Instruction issue phase (except SIMD)\vspace{10pt}
   \item Standard (and future, and custom) opcodes now parallel\vspace{10pt}
  \end{itemize}
  Notes:\vspace{10pt}
   \begin{itemize}
   \item LOAD/STORE (inc. C.LD and C.ST, LDX: everything)\vspace{10pt}
   \item All ALU ops (soft / hybrid / full HW, on per-op basis)\vspace{10pt}
   \item All branches become predication targets (C.FNE added)\vspace{10pt}
  \end{itemize}
}

\begin{frame}[fragile]
\frametitle{ADD pseudocode (or trap, or actual hardware loop)}

\begin{semiverbatim}
function op_add(rd, rs1, rs2, predr) \{ # add not PADD!
  int i, id=0, irs1=0, irs2=0;
  for (i=0; i < MIN(VL, vectorlen[rd]); i++)
    if (ireg[predr] & 1<<i) # predication uses intregs
       ireg[rd+id] <= ireg[rs1+irs1] + ireg[rs2+irs2];
    # now increment idxs: src/dest all vec/scalar
    if (reg_is_vectorised[rd]) \{ id += 1; \}
    if (reg_is_vectorised[rs1]) \{ irs1 += 1; \}
    if (reg_is_vectorised[rs2]) \{ irs2 += 1; \}
\}
\end{semiverbatim}
   \begin{itemize}
   \item Scalar-scalar and scalar-vector and vector-vector now all in one
   \item OoO may choose to push ADDs into instr. queue (v. busy!)
  \end{itemize}
\end{frame}

\frame{\frametitle{How are SIMD Instructions Vectorised?}

 \begin{itemize}
   \item SIMD ALU(s) primarily unchanged\vspace{10pt}
   \item Predication is added to each SIMD element (NO ZEROING!)\vspace{10pt}
   \item End of Vector enables predication (NO ZEROING!)\vspace{10pt}
  \end{itemize}
  Considerations:\vspace{10pt}
   \begin{itemize}
   \item Many SIMD ALUs possible (parallel execution)\vspace{10pt}
   \item Very long SIMD ALUs could waste die area (short vectors)\vspace{10pt}
   \item Implementor free to choose (API remains the same)\vspace{10pt}
  \end{itemize}
}

\frame{\frametitle{What's the deal / juice / score?}

 \begin{itemize}
   \item Standard Register File(s) overloaded with "vector span"\vspace{10pt}
   \item Element width and type concepts remain same as RVV\vspace{10pt}
   \item CSRs are key-value tables (overlaps allowed)\vspace{10pt}
  \end{itemize}
  Key differences from RVV:\vspace{10pt}
   \begin{itemize}
   \item Predication in INT regs as a BIT field (max VL=XLEN)\vspace{10pt}
   \item Minimum VL must be Num Regs - 1 (all regs single LD/ST)\vspace{10pt}
   \item NO ZEROING: non-predicated elements are skipped\vspace{10pt}
  \end{itemize}
}

\frame{\frametitle{Why are overlaps allowed in Regfiles?}

 \begin{itemize}
   \item Same register(s) can have multiple "interpretations"\vspace{10pt}
   \item xBitManip plus SIMD plus xBitManip = Hi/Lo bitops\vspace{10pt}
   \item (32-bit GREV plus 4-wide 32-bit SIMD plus 32-bit GREVI)\vspace{10pt}
   \item 32-bit op followed by 16-bit op w/ 2x VL, 1/2 predicated\vspace{10pt}
  \end{itemize}
  Note:\vspace{10pt}
   \begin{itemize}
   \item xBitManip reduces O($N^{6}$) SIMD down to O($N^{3}$) \vspace{10pt}
   \item Hi-Performance: Macro-op fusion (more pipeline stages?)\vspace{10pt}
  \end{itemize}
}


\frame{\frametitle{Why no Zeroing (place zeros in non-predicated elements)?}

 \begin{itemize}
   \item Zeroing is an implementation optimisation favouring OoO\vspace{10pt}
   \item Simple implementations may skip non-predicated operations\vspace{10pt}
   \item Simple implementations explicitly have to destroy data\vspace{10pt}
   \item Complex implementations may use reg-renames to save power\vspace{10pt}
  \end{itemize}
  Considerations:\vspace{10pt}
  \begin{itemize}
   \item Complex not really impacted, Simple impacted a LOT\vspace{10pt}
   \item Please don't use Vectors for "security" (use Sec-Ext)\vspace{10pt}
  \end{itemize}
}


\frame{\frametitle{slide}

 \begin{itemize}
   \item \vspace{10pt}
  \end{itemize}
  Considerations:\vspace{10pt}
  \begin{itemize}
   \item \vspace{10pt}
  \end{itemize}
}


\frame{\frametitle{slide}

 \begin{itemize}
   \item \vspace{10pt}
  \end{itemize}
  Considerations:\vspace{10pt}
  \begin{itemize}
   \item \vspace{10pt}
  \end{itemize}
}


\frame{\frametitle{slide}

 \begin{itemize}
   \item \vspace{10pt}
  \end{itemize}
  Considerations:\vspace{10pt}
  \begin{itemize}
   \item \vspace{10pt}
  \end{itemize}
}


\frame{\frametitle{Including a plot}
 \begin{center}
%  \includegraphics[height=2in]{dental.ps}\\
  {\bf \red Dental trajectories for 27 children:}
 \end{center}
}

\frame{\frametitle{Creating .pdf slides in WinEdt}

 \begin{itemize}
   \item LaTeX [Shift-Control-L]\vspace{10pt}
   \item dvi2pdf [click the button]\vspace{24pt}
  \end{itemize}
  To print 4 slides per page in acrobat click\vspace{10pt}
   \begin{itemize}
   \item File/print/properties\vspace{10pt}
   \item Change ``pages per sheet'' to 4\vspace{10pt}
  \end{itemize}
}

\frame{
  \begin{center}
    {\Huge \red The end}
  \end{center}
}


\end{document}
