\documentclass[slidestop]{beamer}
\usepackage{beamerthemesplit}
\usepackage{graphics}
\usepackage{pstricks}

\graphicspath{{./}}

\title{The Libre-SOC Hybrid 3D CPU}
\author{Luke Kenneth Casson Leighton}


\begin{document}

\frame{
   \begin{center}
    \huge{The Libre-SOC Hybrid 3D CPU}\\
    \vspace{32pt}
    \Large{Draft SVP64 in-place Matrix Multiply}\\
    \Large{and FFT / DCT for the Power ISA}\\
    \vspace{24pt}
    \Large{OpenPOWER Summit 2021}\\
    \vspace{16pt}
    \large{Sponsored by NLnet's PET Programme}\\
    \vspace{6pt}
    \large{28th Oct 2021}
  \end{center}
}


\frame{\frametitle{}

\vspace{15pt}

 \begin{itemize}
   \item \vspace{15pt}
   \item \vspace{15pt}
   \item \vspace{15pt}
  \end{itemize}
}

\frame{\frametitle{Overview of Libre-SOC goals}

\vspace{8pt}

 \begin{itemize}
   \item To create power-efficient mass-volume products\vspace{8pt}
   \item To leverage the OpenPOWER ecosystem to do so\vspace{8pt}
   \item To be entirely transparent for Security reasons\vspace{8pt}
   \item To empower businesses to bring Secure transparent\\
         mass-volume products to market\vspace{8pt}
   \item Mass-volume end-user products need 3D, Video, Audio 
         \textbf{therefore we require small-size Matrices (3x3 but not with
         75\% utilisation, and 4x4) and the core strategic parts
         of A/V CODECs and that means DCT and FFT.}
         Anything else is a bonus (NTT with Galois Field bitmanip)
  \end{itemize}
}

\frame{\frametitle{Overview of SVP64 goals}

\vspace{15pt}

 \begin{itemize}
   \item High performance and high performance/watt\vspace{10pt}
   \item Reduced code density (reduced I-Cache usage)\\
         https://arxiv.org/abs/2002.10143 - 3.5x power reduction\vspace{8pt}
   \item Remain accessible for assembler writers and compilers alike\vspace{10pt}
   \item Introduce true Vectorisation to the Power ISA\\
         (VSX is Packed SIMD)\vspace{8pt}
   \item Be adopted via the external OPF ISA WG RFC process\\
         (not: be a non-official custom extension. proprietary\\
         custom extensions conflict with mass-volume adoption)\vspace{10pt}
  \end{itemize}
}



\begin{frame}[fragile]
\frametitle{Reminder of Simple-V}

\begin{semiverbatim}
https://libre-soc.org/openpower/sv/overview/
Greatly simplified (like x86 "REP" instruction):

  for (i = 0; i < VL; i++)
       GPR[RT+i] <= GPR[RA+i] + GPR[RB+i];

function op\_add(RT, RA, RB, predr) # add not VADD!
  int i, id=0, irs1=0, irs2=0;
  for (i = 0; i < VL; i++)
    if (GPR[predr] & 1<<i) # predication
       GPR[RT+id] <= GPR[RA+irs1] + GPR[RB+irs2];
    if (reg\_is\_vectorised[RT])  \{ id += 1; \}
    if (reg\_is\_vectorised[RA])  \{ irs1 += 1; \}
    if (reg\_is\_vectorised[RB])  \{ irs2 += 1; \}
\end{semiverbatim}

\end{frame}


\begin{frame}[fragile]
\frametitle{SVP64 REMAP system}

\begin{semiverbatim}
Register offsets are "REMAP"ed through a Hardware FSM
https://libre-soc.org/openpower/sv/remap/
remarkably similar to ZOLC
https://www.researchgate.net/publication/224647569

function op\_add(RT, RA, rs2, predr) # add not VADD!
  int i, id=0, irs1=0, irs2=0;
  for (i = 0; i < VL; i++)
    if (GPR[predr] & 1<<i) # predication
       GPR[RT+REMAP(id)] <= GPR[RA+REMAP(irs1)] +
                           GPR[rs2+REMAP(irs2)];
    if (reg\_is\_vectorised[RT])  \{ id += 1; \}
    if (reg\_is\_vectorised[RA])  \{ irs1 += 1; \}
    if (reg\_is\_vectorised[s2])  \{ irs2 += 1; \}
\end{semiverbatim}

\end{frame}

\begin{frame}[fragile]
\frametitle{Matrix Multiply: Basics}

\begin{semiverbatim}
(a00 a01 a02  x (b00 b01   =   (c00 c01
 a10 a11 a12)    b10 b11        c10 c11)  = ...
                 b20 b21)

(a00*b00 + a01*b10 + a02*b20 a00*b01 + a01*b11 + a02*b21
 a10*b00 + a11*b10 + a12*b20 a10*b01 + a11*b11 + a12*b21)

 (b00 b01    x (a00 a01 a02  =   (c00 c01 c02
  b10 b11       a10 a11 a12)      c10 c11 c12
  b20 b21)                        c20 c21 c22)  = ...

(b00*a00 + b01*a10  b00*a01 + b01*a11  b00*a02 + b01*a12
 b10*a00 + b11*a10  b10*a01 + b11*a11  b10*a02 + b11*a12
 b20*a00 + b21*a10  b20*a01 + b21*a11  b20*a02 + b21*a12)

\end{semiverbatim}

\end{frame}


\begin{frame}[fragile]
\frametitle{Matrix Multiply: naive, with python for-loops}

\begin{semiverbatim}
result = [] # final result
for i in range(len(A)):

  row = [] # the new row in new matrix
  for j in range(len(B[0])):

    product = 0 # the new element in the new row
    for v in range(len(A[i])):
        product += A[i][v] * B[v][j]
    row.append(product) # add sum of product to new row

  result.append(row) # add new row into final result
\end{semiverbatim}

\end{frame}

\begin{frame}[fragile]
\frametitle{Matrix Multiply: suitable for Hardware scheduling}

\begin{semiverbatim}
Unsuitable: creates massive Read-After-Write chains

for i in range(len(A)):
  for j in range(len(B[0])):
    for v in range(len(A[i])):
      product[i][j] += A[i][v] * B[v][j]

Suitable: can be parallelised / pipelined. RaW avoided

for i in range(len(A)):
  for v in range(len(A[i])):    # swapped
    for j in range(len(B[0])):  # with this
      product[i][j] += A[i][v] * B[v][j]

\end{semiverbatim}

\end{frame}


\frame{\frametitle{Matrix Multiply: Generalise but Specialise}

 \begin{itemize}
   \item Why not make a general-purpose nested "Loop" system?\\
        - Other uses (algorithms) beyond Matrix Multiplication\\
        - Allow any arbitrary-sized loops\\
        - Allow any permutation of nesting\\
        - Allow reversing per-dimension
   \item Specialise by making Matrix Multiply "setup" quick/easy\\
        - two 32-bit instructions to set up A, B, C sizes\\
        - one 64-bit SVP64 FMAC instruction (hot-loop).\\
        - Nothing else needed.  Saves on I-Cache
   \item Hardware turns out to be near-identical to ZOLC\\
        https://opencores.org/projects/hwlu\\
        https://libre-soc.org/openpower/sv/remap/
   \item Concept is actually borrowed from Aspex Array-String Processor
         1D/2D/3D Memory DMA "reordering" Engine (except applied to
         the register file)
  \end{itemize}
}

\begin{frame}[fragile]
\frametitle{Matrix Multiply: unit test / example}

\begin{semiverbatim}
  94 def test_sv_remap2(self):
  95     lst = ["svshape 5, 4, 3, 0, 0",
  96            "svremap 0b11111, 1, 2, 3, 0, 0, 0, 0",
  97            "sv.fmadds 0.v, 8.v, 16.v, 0.v"
  98           ]
  99             REMAP fmadds FRT, FRA, FRC, FRB

svshape 5, 4, 3, 0, 0 => A: 3x5 B: 3x4
                      => C: 3x3
svremap (enable) (F)RS, (F)RT, (F)RA, (F)RB, (F)RC
sv.fmadds: uses fp0 as accumulator
      product[i][j] += A[i][v] * B[v][j]
\end{semiverbatim}

\end{frame}

\frame{\frametitle{Matrix Multiply: Ehm that's all Folks}

\vspace{6pt}

 \begin{itemize}
   \item Really is that straightforward: no actual Vector ops\\
        - Does not dictate or limit micro-architectural detail\\
        - Issues Scalar FMACs into existing back-end hardware\\
        - Can use any 4-operand instruction (GF, INT, Bitmanip)\\
        - No Power-2 limits. Any operand width (8/16/32/64)\vspace{8pt}
   \item Limited to 127 scalar ops and in-place registers.  Future?\\
        - https://arxiv.org/abs/2002.10143 CISC-like load-and-inc\\
        - Auto-load/store (tagged) registers, keeps RISC ISA\\
        - Extend to memory-based arbitrary NxN matrix sizes\\
        - Still power-efficient: no I-cache usage during FMAC issue\vspace{8pt}
   \item Future can be investigated as part of EUR 22.6m EU Grant\\
        https://libre-soc.org/SEP-210803722-Libre-SOC-8-core/\vspace{15pt}
  \end{itemize}
}


\frame{\frametitle{DCT / FFT / DFT / NTT: what if we could REMAP?}

\vspace{6pt}

 \begin{itemize}
   \item Can we create a REMAP Schedule for FFT (etc)? YES\\
        - More complicated than Matrix Schedules but same principle\\
        - Again: issues Scalar instructions into back-end micro-arch\\
        - Requires 5-operand (3-in, 2-out) new Scalar Instructions\\
        - Any operand width (8/16/32/64)\vspace{8pt}
   \item Limited to in-place registers and Power-of-Two.  Future?\\
        - Again: CISC-like auto-load/store-and-increment\\
        - https://arxiv.org/abs/2002.10143\\
        - Again: still power-efficient (no I-Cache usage in loops)\vspace{8pt}
   \item Again: can be investigated as part of EUR 22.6m EU Grant\\
        https://libre-soc.org/SEP-210803722-Libre-SOC-8-core/\vspace{15pt}
  \end{itemize}
}


\frame{\frametitle{Discrete Cosine Transform (DCT): Basics}

 \begin{itemize}
   \item Standard DCT Schedule (messy, impossible for SIMD)
   \item Output is in bit-reversed order\\
         0b000 = 0b000 (in: 0   out: 0)\\
         0b001 = 0b100 (in: 1   out: 4) ...\\
         0b110 = 0b011 (in: 6   out: 3)\\
         0b111 = 0b111 (in: 7   out: 7)
  \end{itemize}

\begin{center}
\includegraphics[width=0.75\textwidth]{plonka_dct1.png}
\end{center}

}

\frame{\frametitle{Fast Fourier Transform (FFT/DFT): Butterfly Basics}

 \begin{itemize}
   \item Standard Butterfly Schedule (again: messy, but less so)
   \item Output, again, is in bit-reversed order
  \end{itemize}

\begin{center}
\includegraphics[width=0.70\textwidth]{fft_butterfly.png}
\end{center}

}

\frame{\frametitle{FFT: 3-in, 2-out butterfly}

 \begin{itemize}
   \item One multiply (by coefficient), one add, one subtract
   \item inputs: X[0] X[1] C(oeff) outputs: X[0] X[1]
  \end{itemize}

\begin{center}
\includegraphics[width=0.90\textwidth]{2-point-dft.png}
\end{center}

}


\begin{frame}[fragile]
\frametitle{DFT: Project Nayuki Radix-2 Cooley-Tukey DIT (1)}

\begin{semiverbatim}
coef = (2 if inverse else -2) * cmath.pi / n
exptable = [cmath.rect(1, i*coef) for i in range(n // 2)]
vec = [vec[reverse_bits(i, levels)] for i in range(n)]
size = 2
while size <= n:
    halfsize, tablestep = size // 2, n // size
    for i in range(0, n, size):
        k = 0
        for j in range(i, i + halfsize):
            temp = vec[j + halfsize] * exptable[k]
            vec[j + halfsize] = vec[j] - temp
            vec[j] += temp
            k += tablestep
    size *= 2
\end{semiverbatim}

\end{frame}


\begin{frame}[fragile]
\frametitle{DFT: Project Nayuki Radix-2 Cooley-Tukey DIT (2)}

\begin{semiverbatim}
coef = (2 if inverse else -2) * cmath.pi / n
exptable = [cmath.rect(1, i*coef) for i in range(n // 2)]
vec = [vec[reverse_bits(i, levels)] for i in range(n)]
size = 2
while size <= n:
    hs, tablestep = size // 2, n // size
    for i in range(0, n, size):
        k = 0
        for j in range(i, i+hs):
            # Twin-Butterfly 3-in 2-out: one instruction
            C = exptable[k]
            vec[j+hs], vec[j] = 2B(vec[j+hs], vec[j], C)
            k += tablestep
    size *= 2
\end{semiverbatim}

\end{frame}

\begin{frame}[fragile]
\frametitle{DFT: Project Nayuki Radix-2 Cooley-Tukey DIT (3)}

 \begin{itemize}
    \item What if the Triple Loop could be done with REMAP?
    \item Register Offsets j, j+hs, k created automatically?
    \item Only one actual inner loop instruction (Twin-butterfly)
    \item 3-in (X0/X1/C) 2-out (X0/X1) allows for in-place FFT
    \item Hardware FSM (like ZOLC) creates offset triplet\\
        - Technically not that hard to implement (for Radix-2)\\
        - Exact same principle as Triple-loop for Matrices
  \end{itemize}

\begin{semiverbatim}
for j,k,hs in REMAP_TRIPLE_LOOP_GENERATOR():
            # Twin-Butterfly 3-in 2-out: one instruction
            C = exptable[k]
            vec[j+hs], vec[j] = 2B(vec[j+hs], vec[j], C)
\end{semiverbatim}

\end{frame}

\frame{\frametitle{DCT: pre-arrange (pre-load) data}

 \begin{itemize}
   \item Arrange input data such that output falls into place
   \item (another) Twin 3-in 2-out Mul-Add in-place instruction
  \end{itemize}

\begin{center}
\includegraphics[width=0.70\textwidth]{dct_butterfly.png}
\end{center}

}


\frame{\frametitle{FFT (Complex numbers) and DCT coefficients?}

 \begin{itemize}
   \item Problem (DCT): DCT Cosine Coefficients change (cos + 0.5)
         depending on the layer.  Cannot do as single instruction
   \item Problem (FFT): Complex number butterfly multiplication involves
         4 multiplies.  Cannot do in-place as single instruction\vspace{12pt}
   \item Solution: "Vertical-First" Vectors (Mitch Alsup 66000 ISA)\vspace{12pt}
   \item Understanding of SVP64 "Vertical-First" 30min video
         https://youtube.com/watch?v=fn2KJvWyBKg
   \item Basically involves stepping "vertically" through instructions
         then ("stepping") to the next offset (REMAP), loop with bc
   \item Horizontal-first: run through the entire REMAP schedule on a single
         instruction before repeating looping on the next

  \end{itemize}
}

\frame{\frametitle{Summary}

 \begin{itemize}
   \item Goal is to create a mass-volume low-power embedded SoC suitable
         for use in netbooks, chromebooks, tablets, smartphones, IoT SBCs.
   \item This means a computational focus on 3D and Audio/Video.\\
         - Critical not to waste 75\% of Power-2 SIMD Lanes on 3x3
   \item Reducing core work to a one-instruction hot-loop inherently
         reduces power consumption because the I-Cache is 100\% idle.
   \item REMAP system completely independent from the instructions it
         REMAPs. Applies to future scalar ops (GF, Bitmanip)
   \item Future versions involve proper Zero-Overhead Loop-Control
         and hidden "tags" to automatically perform CISC-like
         auto-load/store-and-inc (for much larger data sets)
   \item Please help contribute: it's your Open Power ISA too.

  \end{itemize}
}


\frame{
  \begin{center}
    {\Huge The end\vspace{10pt}\\
		   Thank you\vspace{10pt}\\
		   Questions?\vspace{10pt}
	}
  \end{center}

  \begin{itemize}
	\item Discussion: Libre-SOC-ISA mailing list\\
        http://lists.libre-soc.org/mailman/listinfo/libre-soc-isa
	\item Libera IRC \#libre-soc
	\item http://libre-soc.org/
	\item http://nlnet.nl/PET\\
    https://www.ngi.eu/ngi-projects/ngi-pointer/
  \end{itemize}
}


\end{document}
