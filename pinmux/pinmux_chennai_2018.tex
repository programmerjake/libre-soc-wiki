\documentclass[slidestop]{beamer}
\usepackage{beamerthemesplit}
\usepackage{graphics}
\usepackage{pstricks}

\title{Pin Multiplexer}
\author{Rishabh Jain}
\author{Luke Kenneth Casson Leighton}


\begin{document}

\frame{
   \begin{center}
    \huge{Pin Multiplexer}\\
    \vspace{32pt}
    \Large{Auto-generating documentation, code \\
			and resources for a Pinmux}\\
    \vspace{24pt}
    \Large{[proposed for] Chennai 9th RISC-V Workshop}\\
    \vspace{16pt}
    \large{\today}
  \end{center}
}


\frame{\frametitle{Credits and Acknowledgements}

 \begin{itemize}
   \item TODO\vspace{10pt}
  \end{itemize}
}


\frame{\frametitle{Glossary}

 \begin{itemize}
   \item GPIO: general-purpose reconfigureable I/O (Input/Output).
   \item Pin: an I/O pad.  May be driven (input) or may drive (output).
   \item FN: term for a single-wire "function", such as UART\_TX,
	     I2C\_SDA, SDMMC\_D0 etc.  may be an input, output or both
		 (bi-directional case: two wires are always allocated, one
		  for input to the function and one for output from the function).
   \item Bus: a group of bi-directional functions (SDMMC D0 to D3)
         where the direction is ganged and under the Bus's control
   \item Input Priority Muxer: a multiplexer with N selector
		 wires and N associated inputs.  The lowest (highest?) indexed
		 "selector" enabled results in its 
		 input being routed to the output.
   \item Output Demuxer: a one-to-many "redirector" where a single
	     input is "routed" to any one output, based
	     on a selector.
  \end{itemize}
}


\frame{\frametitle{Why, How and What is a Pinmux?}

 \begin{itemize}
   \item Why? To save cost, increase yield, and to target multiple
         markets with the same design, thereby increasing uptake
         and consequently taking advantage of volume pricing.\vspace{4pt}
         \\
         Summary: it's all about making more money!\vspace{4pt}
   \item How? By multiplexing many more functions (100 to 1,200) than there
         are actual available pins (48 to 500), the required chip package
	     is far less costly and the chip more desirable\vspace{4pt}
   \item What? A many-to-many dynamically-configureable router of
         I/O functions to I/O pins\vspace{4pt}
   \item \bf{Note: actual muxing is deceptively simple, but like
         a DRAM cell it's actually about the ancillaries / extras}
  \end{itemize}
}


\frame{\frametitle{Associated Extras}

 \begin{itemize}
   \item Design Specification
   \item Scenario analysis (whether the chip will fit "markets")
   \item Documentation: Summary sheet, Technical Reference Manual.
   \item Test suites
   \item Control Interface (AXI4 / Wishbone / TileLink / other)
   \item Simulation
   \item Linux kernel drivers, DTB, libopencm3, Arduino libraries etc.
  \end{itemize}
 Example context:
 \begin{itemize}
   \item Shakti M-Class has 160 pins with a 99.5\% full 4-way mux
   \item Almost 640-way routing, 6 "scenarios" (7th TBD),
         100+ page Manual needed,
         \bf{17,500 lines of auto-generated code}
  \end{itemize}
}


\frame{
  \vspace{30pt}
  \begin{center}
    {\Huge 
		   ALL of these\vspace{20pt}\\
		   can be\vspace{20pt}\\
		   auto-generated\vspace{30pt}
	}
	\\
	(i.e. it would be insanely costly to do them by hand)
  \end{center}
  
}

\frame{\frametitle{Reduce workload, reduce duplication, reduce risk and cost}

 \begin{itemize}
   \item Auto-generate everything: documentation, code, libraries etc.
	     \vspace{10pt}
   \item Standardise: similar to PLIC, propose GPIO and Pinmux
	     \vspace{10pt}
   \item Standardise format of configuration registers:
	     saves code duplication effort (multiple software environments)
		 \vspace{10pt}
   \item Add support for multiple code formats: Chisel3 (SiFive IOF),
	     BSV (Bluespec), Verilog, VHDL, MyHDL.
		 \vspace{10pt}
   \item Multiple auto-generated code-formats permits cross-validation:\\
	     auto-generated test suite in one HDL can validate a muxer
	     generated for a different target HDL.
   		 \vspace{10pt}
  \end{itemize}
}

\frame{\frametitle{Muxer cases to handle (One/Many to One/Many) etc.}

 \begin{itemize}
   \item One FN outputs to Many Pins: no problem\\
	     (weird configuration by end-user, but no damage to ASIC)
   \item One Pin to Many FN inputs: no problem\\
         (weird configuration by end-user, but no damage to ASIC)
   \item Many Pins to One FN input: {\bf Priority Mux needed}\\
	     No priority mux: Pin1 = HI, Pin0 = LO, ASIC is damaged
   \item Many FN outputs simultaneously to one Pin: {\bf does not occur}\\
	     (not desirable and not possible, as part of the pinmux design)
   \item Some FNs (I2C\_SDA, SD\_D0..3) are I/O Buses\\
	     Bi-directional control of the Pin must be handed to the
	     FN
   \item Nice to have: Bus sets pintype, signal strength etc.\\
	     e.g. selecting SD/MMC doesn't need manual pin-config.\\
	     \bf{caveat: get that wrong and the ASIC can't be sold}
  \end{itemize}
}


\frame{\frametitle{Pin Configuration, input and output}

 In/out: {\bf Note: these all require multiplexing }
 \begin{itemize}
   \item Output-Enable (aka Input disable): switches pad to In or Out
   \item Output (actually an input wire controlling pin's level, HI/LO)
   \item Input (actually an output wire set based on pin's driven level)
 \end{itemize}
 Characteristics: {\bf Note: these do not require multiplexing }
 \begin{itemize}
   \item Output current level: 10mA / 20mA / 30mA / 40mA
   \item Input hysteresis: low / middle / high. Stops signal noise
   \item Pin characteristics: CMOS Push-Push / Open-Drain
   \item Pull-up enable: built-in 10k (50k?) resistor
   \item Pull-down enable: built-in 10k (50k?) resistor
   \item Muxing and IRQ Edge-detection not part of the I/O pin
  \end{itemize}
}


\frame{\frametitle{Standard GPIO 4-way in/out Mux and I/O pad}
 \begin{center}
  \includegraphics[height=2.5in]{../shakti/m_class/mygpiomux.jpg}\\
  {\bf 4-in, 4-out, pullup/down, hysteresis, edge-detection (EINT)}
 \end{center}
}

\frame{\frametitle{Pin Configuration, input and output}

 \begin{itemize}
   \item Standard Mux design {\bf cannot deal with many-to-one inputs}\\
	     (SiFive IOF source code from Freedom U310 cannot, either)
	     \vspace{4pt}
   \item I/O pad configuration conflated with Muxer conflated with GPIO
         conflated with EINT
         	     \vspace{4pt}
 \end{itemize}
   {\bf IMPORTANT to separate all of these out:
            	     \vspace{4pt}}
 \begin{itemize}
   \item EINTs to be totally separate FNs. managed by RISC-V PLIC\\
         (If every GPIO was an EINT it would mean 100+ IRQs)
            	     \vspace{4pt}
   \item GPIO In/Out/Direction treated just like any other FN\\
	     (but happen to have AXI4 - or other - memory-mapping)
            	     \vspace{4pt}
   \item Pad configuration separated and given one-to-one Registers\\
	     (SRAMs set by AXI4 to control mux, pullup, current etc.)
 \end{itemize}
}

\frame{\frametitle{Register-to-pad "control" settings}
 \begin{center}
  \includegraphics[height=2.5in]{reg_gpio_cap_ctrl.jpg}\\
  {\bf pullup/down, hysteresis, current, edge-detection}
 \end{center}
}


\frame{\frametitle{In/Out muxing, direction control}
 \begin{center}
  \includegraphics[height=2.5in]{reg_gpio_fn_ctrl.jpg}\\
  {\bf Note: function can control I/O direction}
 \end{center}
}


\frame{\frametitle{Simplified I/O pad Block Diagram}
 \begin{center}
  \includegraphics[height=2.5in]{reg_gpio_pinblock.jpg}\\
  {\bf 3 wires: IN, OUT, OUTEN (also = !INEN) }
 \end{center}
}


\frame{\frametitle{Output (and OUTEN) Wiring. 2 pins, 2 GPIO, 2 Fns}
 \begin{center}
  \includegraphics[height=2.5in]{reg_gpio_out_wiring.jpg}\\
  {\bf Reg0 for Pin0, Reg1 for Pin1, Output and OUTEN same mux }
 \end{center}
}


\frame{\frametitle{Input Selection and Priority Muxing}
 \begin{center}
  \includegraphics[height=0.75in]{reg_gpio_comparator.jpg}\\
  {\bf Muxer enables input selection}\\
  \vspace{10pt}
  \includegraphics[height=1.25in]{reg_gpio_in_prioritymux.jpg}\\
  {\bf However multiple inputs must be prioritised }
 \end{center}
}


\frame{\frametitle{Input Mux Wiring}
 \begin{center}
  \includegraphics[height=2.5in]{reg_gpio_in_wiring.jpg}\\
  {\bf Pin Mux selection vals NOT same as FN selection vals}
 \end{center}
}


\frame{\frametitle{Summary}

 \begin{itemize}
   \item TODO
  \end{itemize}
}


\frame{
  \begin{center}
    {\Huge The end\vspace{20pt}\\
		   Thank you\vspace{20pt}\\
		   Questions?\vspace{20pt}
	}
  \end{center}
  
  \begin{itemize}
	\item http://libre-riscv.org/shakti/m\_class/pinmux/
  \end{itemize}
}


\end{document}
