\documentclass[slidestop]{beamer}
\usepackage{beamerthemesplit}
\usepackage{graphics}
\usepackage{pstricks}

\title{SIMD}
\author{Rishabh Jain}
\author{Luke Kenneth Casson Leighton}


\begin{document}

\frame{
   \begin{center}
    \huge{Pin Multiplexer}\\
    \vspace{32pt}
    \Large{Auto-generating documentation, code \\
			and resources for a Pinmux}\\
    \vspace{24pt}
    \Large{[proposed for] Chennai 9th RISC-V Workshop}\\
    \vspace{16pt}
    \large{\today}
  \end{center}
}


\frame{\frametitle{Credits and Acknowledgements}

 \begin{itemize}
   \item TODO\vspace{10pt}
  \end{itemize}
}


\frame{\frametitle{Glossary}

 \begin{itemize}
   \item Pin: an I/O pad.  May be driven (input) or may drive (output).
   \item FN: term for a single-wire "function", such as UART\_TX,
	     I2C\_SDA, SDMMC\_D0 etc.  may be an input, output or both
		 (bi-directional case: two wires are always allocated, one
		  for input to the function and one for output from the function).
   \item Input Priority Muxer: a multiplexer that has N selector
		 wires and N inputs, where the lowest (or highest) indexed
		 "selector" that is enabled results in its corresponding
		 input being routed to the output.
   \item Output Demuxer: a one-to-many "redirector" where a single
	     input is "routed" to any one of a number of outputs, based
	     on a selection address.
   \item GPIO: general-purpose reconfigureable I/O (Input/Output).
  \end{itemize}
}


\frame{\frametitle{Why, How and What is a Pinmux?}

 \begin{itemize}
   \item Why? To save cost, increase yield, and to target multiple
         markets with the same design, thereby increasing uptake
         and consequently taking advantage of volume pricing.\vspace{4pt}
         \\
         Summary: it's all about making more money!\vspace{4pt}
   \item How? By multiplexing many more functions (100 to 1,200) than there
         are actual available pins (48 to 500), the required chip package
	     is far less costly and the chip more desirable\vspace{4pt}
   \item What? A many-to-many dynamically-configureable router of
         I/O functions to I/O pins\vspace{4pt}
   \item \bf{Note: actual muxing is deceptively simple, but like
         a DRAM cell it's actually about the ancillaries / extras}
  \end{itemize}
}


\frame{\frametitle{Associated Extras}

 \begin{itemize}
   \item Design Specification
   \item Scenario analysis (whether the chip will fit "markets")
   \item Documentation: Summary sheet, Technical Reference Manual.
   \item Test suites
   \item Control Interface
   \item Simulation
   \item Linux kernel drivers, DTB, libopencm3, Arduino
  \end{itemize}
 Example context:
 \begin{itemize}
   \item Shakti M-Class has 160 pins with a 99.5\% full 4-way mux
   \item Almost 640-way routing, 6 "scenarios" (7th TBD),
         100+ page Manual needed,
         \bf{17,500 lines of auto-generated code}
  \end{itemize}
}


\frame{
  \vspace{30pt}
  \begin{center}
    {\Huge 
		   ALL of these\vspace{20pt}\\
		   can be\vspace{20pt}\\
		   auto-generated\vspace{30pt}
	}
	\\
	(translation: it would be insanely costly to do them by hand)
  \end{center}
  
}

\frame{\frametitle{Muxer cases to handle}

 \begin{itemize}
   \item Many FN outputs to Many Pins: no problem\\
	     (weird configuration by end-user, but no damage to ASIC)\vspace{10pt}
   \item One Pin to Many FN inputs: no problem\\
         (weird configuration by end-user, but no damage to ASIC)\vspace{10pt}
   \item Many Pins to One FN input: {\bf Priority Mux needed}\\
	     No priority mux: Pin1 = HI, Pin0 = LO, ASIC is damaged\vspace{10pt}
   \item Some FNs (I2C\_SDA, SD\_D0..3) are I/O Buses\\
	     Bi-directional control of the Pin must be handed to the
	     FN\vspace{10pt}
   \item TODO\vspace{10pt}
  \end{itemize}
}


\frame{\frametitle{Standard GPIO 4-way in/out Mux and I/O pad}
 \begin{center}
  \includegraphics[height=2.5in]{../shakti/m_class/mygpiomux.jpg}\\
  {\bf 4-in, 4-out, pullup/down, hysteresis, edge-detection (EINT)}
 \end{center}
}


\frame{\frametitle{Register-to-pad "control" settings}
 \begin{center}
  \includegraphics[height=2.5in]{reg_gpio_cap_ctrl.jpg}\\
  {\bf pullup/down, hysteresis, current, edge-detection}
 \end{center}
}


\frame{\frametitle{In/Out muxing, direction control}
 \begin{center}
  \includegraphics[height=2.5in]{reg_gpio_fn_ctrl.jpg}\\
  {\bf Note: function can control I/O direction}
 \end{center}
}


\frame{\frametitle{Simplified I/O pad Block Diagram}
 \begin{center}
  \includegraphics[height=2.5in]{reg_gpio_pinblock.jpg}\\
  {\bf 3 wires: IN, OUT, OUTEN (also = !INEN) }
 \end{center}
}


\frame{\frametitle{Output (and OUTEN) Wiring. 2 pins, 2 GPIO, 2 Fns}
 \begin{center}
  \includegraphics[height=2.5in]{reg_gpio_out_wiring.jpg}\\
  {\bf Reg0 for Pin0, Reg1 for Pin1, Output and OUTEN same mux }
 \end{center}
}


\frame{\frametitle{Input Selection and Priority Muxing}
 \begin{center}
  \includegraphics[height=0.75in]{reg_gpio_comparator.jpg}\\
  {\bf Muxer enables input selection}\\
  \vspace{10pt}
  \includegraphics[height=1.25in]{reg_gpio_in_prioritymux.jpg}\\
  {\bf However multiple inputs must be prioritised }
 \end{center}
}


\frame{\frametitle{Input Mux Wiring}
 \begin{center}
  \includegraphics[height=2.5in]{reg_gpio_in_wiring.jpg}\\
  {\bf Pin Mux selection vals NOT same as FN selection vals}
 \end{center}
}


\frame{\frametitle{Summary}

 \begin{itemize}
   \item TODO
  \end{itemize}
}


\frame{
  \begin{center}
    {\Huge The end\vspace{20pt}\\
		   Thank you\vspace{20pt}\\
		   Questions?\vspace{20pt}
	}
  \end{center}
  
  \begin{itemize}
	\item http://libre-riscv.org/shakti/m\_class/pinmux/
  \end{itemize}
}


\end{document}
